% generated from JIRA project LVV
% using template at /usr/share/miniconda/envs/docsteady-env/lib/python3.7/site-packages/docsteady/templates/tpnoresult.latex.jinja2.
% using docsteady version 2.4.1
% Please do not edit -- update information in Jira instead
\documentclass[DM,lsstdraft,STR,toc]{lsstdoc}
\usepackage{geometry}
\usepackage{longtable,booktabs}
\usepackage{enumitem}
\usepackage{arydshln}
\usepackage{attachfile}
\usepackage{array}
\usepackage{dashrule}

\newcolumntype{L}[1]{>{\raggedright\let\newline\\\arraybackslash\hspace{0pt}}p{#1}}

\input meta.tex

\newcommand{\attachmentsUrl}{https://github.com/\gitorg/\lsstDocType-\lsstDocNum/blob/\gitref/attachments}
\providecommand{\tightlist}{
  \setlength{\itemsep}{0pt}\setlength{\parskip}{0pt}}

\setcounter{tocdepth}{4}

\begin{document}

\def\milestoneName{Data Management Acceptance Test Campaign, Fall 2023}
\def\milestoneId{}
\def\product{Acceptance}

\setDocCompact{true}

\title{LVV-P106: Data Management Acceptance Test Campaign, Fall 2023 Test Plan }
\setDocRef{\lsstDocType-\lsstDocNum}
\date{ 2024-05-17 }
\author{ Jeffrey Carlin }

% Most recent last
\setDocChangeRecord{
\addtohist{}{2023-07-01}{First draft}{Jeffrey Carlin}
\addtohist{}{2024-04-08}{Test campaign LVV-P106 completed and results approved. DM-40311}{Jeffrey Carlin}
}

\setDocCurator{Jeffrey Carlin}
\setDocUpstreamLocation{\url{https://github.com/lsst-dm/\lsstDocType-\lsstDocNum}}
\setDocUpstreamVersion{\vcsRevision}



\setDocAbstract{
This is the test plan for
\textbf{ Data Management Acceptance Test Campaign, Fall 2023},
an LSST milestone pertaining to the Data Management Subsystem.\\
This document is based on content automatically extracted from the Jira test database on \docDate.
The most recent change to the document repository was on \vcsDate.
}


\maketitle

\section{Introduction}
\label{sect:intro}


\subsection{Objectives}
\label{sect:objectives}

 The primary goal of this DM acceptance test campaign will be to verify
priority 1a DMSR (\citeds{LSE-61}) requirements that have not been verified as
part of prior testing and milestones. Any priority 1b, 2, or 3
requirements that have been completed will also be verified.



\subsection{System Overview}
\label{sect:systemoverview}

 This test campaign is intended to verify that the DM system satisfies at
least half of the priority 1a requirements outlined in the Data
Management System Requirements (DMSR;
\href{https://lse-61.lsst.io/}{LSE-61} ), ensuring that we are
progressing toward readiness for the installation and operation of
LSSTCam. Additional DMSR requirements will be verified in later
Acceptance Test Campaigns.\\[2\baselineskip]\textbf{Applicable
Documents:}\\
\citeds{LSE-61}: Data Management System (DMS) Requirements\\
\citeds{LDM-503} Data Management Test Plan\\
\citeds{LDM-639}: Data Management Acceptance Test
Specification\\[2\baselineskip]Tests in this campaign will use data
products and artifacts from Data Preview 0.2, which consists of DESC
Data Challenge 2 (DC2) simulated data reprocessed using the LSST Science
Pipelines. Additional on-sky data from auxTel imaging campaigns, and
camera test-stand data, will be used when appropriate.


\subsection{Document Overview}
\label{sect:docoverview}

This document was generated from Jira, obtaining the relevant information from the
\href{https://jira.lsstcorp.org/secure/Tests.jspa\#/testPlan/LVV-P106}{LVV-P106}
~Jira Test Plan and related Test Cycles (
\href{https://jira.lsstcorp.org/secure/Tests.jspa\#/testCycle/LVV-C260}{LVV-C260}
).

Section \ref{sect:intro} provides an overview of the test campaign, the system under test (\product{}),
the applicable documentation, and explains how this document is organized.
Section \ref{sect:testplan} provides additional information about the test plan, like for example the configuration
used for this test or related documentation.
Section \ref{sect:personnel} describes the necessary roles and lists the individuals assigned to them.

Section \ref{sect:overview} provides a summary of the test results, including an overview in Table \ref{table:summary},
an overall assessment statement and suggestions for possible improvements.
Section \ref{sect:detailedtestresults} provides detailed results for each step in each test case.

The current status of test plan \href{https://jira.lsstcorp.org/secure/Tests.jspa\#/testPlan/LVV-P106}{LVV-P106} in Jira is \textbf{ Completed }.

\subsection{References}
\label{sect:references}
\renewcommand{\refname}{}
\bibliography{lsst,refs,books,refs_ads,local}


\newpage
\section{Test Plan Details}
\label{sect:testplan}


\subsection{Data Collection}

  Observing is not required for this test campaign.

\subsection{Verification Environment}
\label{sect:hwconf}
  Most testing will be performed using the Rubin Science Platform (RSP)
and the development cluster at the USDF. In particular, we will use
version 26 of the Pipelines for most tests; some tests will use more
recent weekly builds of the Pipelines.




\subsection{Related Documentation}



\subsection{PMCS Activity}

Primavera milestones related to the test campaign:
\begin{itemize}
\item None
\end{itemize}


\newpage
\section{Personnel}
\label{sect:personnel}

The personnel involved in the test campaign is shown in the following table.

{\small
\begin{longtable}{p{3cm}p{3cm}p{3cm}p{6cm}}
\hline
\multicolumn{2}{r}{T. Plan \href{https://jira.lsstcorp.org/secure/Tests.jspa\#/testPlan/LVV-P106}{LVV-P106} owner:} &
\multicolumn{2}{l}{\textbf{ Jeffrey Carlin } }\\\hline
\multicolumn{2}{r}{T. Cycle \href{https://jira.lsstcorp.org/secure/Tests.jspa\#/testCycle/LVV-C260}{LVV-C260} owner:} &
\multicolumn{2}{l}{\textbf{
Jeffrey Carlin }
} \\\hline
\textbf{Test Cases} & \textbf{Assigned to} & \textbf{Executed by} & \textbf{Additional Test Personnel} \\ \hline
\href{https://jira.lsstcorp.org/secure/Tests.jspa#/testCase/LVV-T146}{LVV-T146}
& {\small Leanne Guy } & {\small Leanne Guy } &
\begin{minipage}[]{6cm}
\smallskip
{\small  }
\medskip
\end{minipage}
\\ \hline
\href{https://jira.lsstcorp.org/secure/Tests.jspa#/testCase/LVV-T1240}{LVV-T1240}
& {\small Jim Bosch } & {\small Jeffrey Carlin } &
\begin{minipage}[]{6cm}
\smallskip
{\small  }
\medskip
\end{minipage}
\\ \hline
\href{https://jira.lsstcorp.org/secure/Tests.jspa#/testCase/LVV-T132}{LVV-T132}
& {\small Leanne Guy } & {\small Jeffrey Carlin } &
\begin{minipage}[]{6cm}
\smallskip
{\small  }
\medskip
\end{minipage}
\\ \hline
\href{https://jira.lsstcorp.org/secure/Tests.jspa#/testCase/LVV-T62}{LVV-T62}
& {\small Jim Bosch } & {\small Jeffrey Carlin } &
\begin{minipage}[]{6cm}
\smallskip
{\small  }
\medskip
\end{minipage}
\\ \hline
\href{https://jira.lsstcorp.org/secure/Tests.jspa#/testCase/LVV-T41}{LVV-T41}
& {\small Jim Bosch } & {\small Jeffrey Carlin } &
\begin{minipage}[]{6cm}
\smallskip
{\small  }
\medskip
\end{minipage}
\\ \hline
\href{https://jira.lsstcorp.org/secure/Tests.jspa#/testCase/LVV-T97}{LVV-T97}
& {\small Kian-Tat Lim } & {\small Jeffrey Carlin } &
\begin{minipage}[]{6cm}
\smallskip
{\small  }
\medskip
\end{minipage}
\\ \hline
\href{https://jira.lsstcorp.org/secure/Tests.jspa#/testCase/LVV-T1946}{LVV-T1946}
& {\small Jeffrey Carlin } & {\small Jeffrey Carlin } &
\begin{minipage}[]{6cm}
\smallskip
{\small  }
\medskip
\end{minipage}
\\ \hline
\href{https://jira.lsstcorp.org/secure/Tests.jspa#/testCase/LVV-T1947}{LVV-T1947}
& {\small Jeffrey Carlin } & {\small Jeffrey Carlin } &
\begin{minipage}[]{6cm}
\smallskip
{\small  }
\medskip
\end{minipage}
\\ \hline
\href{https://jira.lsstcorp.org/secure/Tests.jspa#/testCase/LVV-T28}{LVV-T28}
& {\small Colin Slater } & {\small Jeffrey Carlin } &
\begin{minipage}[]{6cm}
\smallskip
{\small  }
\medskip
\end{minipage}
\\ \hline
\href{https://jira.lsstcorp.org/secure/Tests.jspa#/testCase/LVV-T142}{LVV-T142}
& {\small Colin Slater } & {\small Jeffrey Carlin } &
\begin{minipage}[]{6cm}
\smallskip
{\small  }
\medskip
\end{minipage}
\\ \hline
\href{https://jira.lsstcorp.org/secure/Tests.jspa#/testCase/LVV-T1748}{LVV-T1748}
& {\small Jeffrey Carlin } & {\small Jeffrey Carlin } &
\begin{minipage}[]{6cm}
\smallskip
{\small  }
\medskip
\end{minipage}
\\ \hline
\href{https://jira.lsstcorp.org/secure/Tests.jspa#/testCase/LVV-T1759}{LVV-T1759}
& {\small Jeffrey Carlin } & {\small Jeffrey Carlin } &
\begin{minipage}[]{6cm}
\smallskip
{\small  }
\medskip
\end{minipage}
\\ \hline
\href{https://jira.lsstcorp.org/secure/Tests.jspa#/testCase/LVV-T1758}{LVV-T1758}
& {\small Jeffrey Carlin } & {\small Jeffrey Carlin } &
\begin{minipage}[]{6cm}
\smallskip
{\small  }
\medskip
\end{minipage}
\\ \hline
\href{https://jira.lsstcorp.org/secure/Tests.jspa#/testCase/LVV-T149}{LVV-T149}
& {\small Leanne Guy } & {\small Jeffrey Carlin } &
\begin{minipage}[]{6cm}
\smallskip
{\small  }
\medskip
\end{minipage}
\\ \hline
\href{https://jira.lsstcorp.org/secure/Tests.jspa#/testCase/LVV-T40}{LVV-T40}
& {\small Jeffrey Carlin } & {\small Jeffrey Carlin } &
\begin{minipage}[]{6cm}
\smallskip
{\small  }
\medskip
\end{minipage}
\\ \hline
\end{longtable}
}

\newpage

\section{Test Campaign Overview}
\label{sect:overview}

\subsection{Summary}
\label{sect:summarytable}

{\small
\begin{longtable}{p{2cm}cp{2.3cm}p{8.6cm}p{2.3cm}}
\toprule
\multicolumn{2}{r}{ T. Plan \href{https://jira.lsstcorp.org/secure/Tests.jspa\#/testPlan/LVV-P106}{LVV-P106}:} &
\multicolumn{2}{p{10.9cm}}{\textbf{ Data Management Acceptance Test Campaign, Fall 2023 }} & Completed \\\hline
\multicolumn{2}{r}{ T. Cycle \href{https://jira.lsstcorp.org/secure/Tests.jspa\#/testCycle/LVV-C260}{LVV-C260}:} &
\multicolumn{2}{p{10.9cm}}{\textbf{ Data Management Acceptance Test Campaign, Fall 2023 }} & Done \\\hline
\textbf{Test Cases} &  \textbf{Ver.}  \\\toprule
\href{https://jira.lsstcorp.org/secure/Tests.jspa#/testCase/LVV-T146}{LVV-T146}
&  1
\\
\href{https://jira.lsstcorp.org/secure/Tests.jspa#/testCase/LVV-T1240}{LVV-T1240}
&  1
\\
\href{https://jira.lsstcorp.org/secure/Tests.jspa#/testCase/LVV-T132}{LVV-T132}
&  1
\\
\href{https://jira.lsstcorp.org/secure/Tests.jspa#/testCase/LVV-T62}{LVV-T62}
&  2
\\
\href{https://jira.lsstcorp.org/secure/Tests.jspa#/testCase/LVV-T41}{LVV-T41}
&  1
\\
\href{https://jira.lsstcorp.org/secure/Tests.jspa#/testCase/LVV-T97}{LVV-T97}
&  1
\\
\href{https://jira.lsstcorp.org/secure/Tests.jspa#/testCase/LVV-T1946}{LVV-T1946}
&  1
\\
\href{https://jira.lsstcorp.org/secure/Tests.jspa#/testCase/LVV-T1947}{LVV-T1947}
&  1
\\
\href{https://jira.lsstcorp.org/secure/Tests.jspa#/testCase/LVV-T28}{LVV-T28}
&  1
\\
\href{https://jira.lsstcorp.org/secure/Tests.jspa#/testCase/LVV-T142}{LVV-T142}
&  1
\\
\href{https://jira.lsstcorp.org/secure/Tests.jspa#/testCase/LVV-T1748}{LVV-T1748}
&  1
\\
\href{https://jira.lsstcorp.org/secure/Tests.jspa#/testCase/LVV-T1759}{LVV-T1759}
&  1
\\
\href{https://jira.lsstcorp.org/secure/Tests.jspa#/testCase/LVV-T1758}{LVV-T1758}
&  1
\\
\href{https://jira.lsstcorp.org/secure/Tests.jspa#/testCase/LVV-T149}{LVV-T149}
&  1
\\
\href{https://jira.lsstcorp.org/secure/Tests.jspa#/testCase/LVV-T40}{LVV-T40}
&  1
\\
\\\hline
\caption{Test Campaign Summary}
\label{table:summary}
\end{longtable}
}

\subsection{Overall Assessment}
\label{sect:overallassessment}

In this test campaign, we have successfully verified 9 unique
requirements from \citeds{LSE-61} via the execution of 15 Test Cases (all of
which passed). Of these requirements, 6 are of priority 1a, and 3 are
priority 1b. The set of requirements tested in this campaign mostly
cover aspects of the DM system related to the software pipelines and the
facilities for data processing and handling. These tests were performed
at the US Data Facility (USDF) using precursor HSC and Auxtel data.

\subsection{Recommended Improvements}
\label{sect:recommendations}

\newpage
\section{Detailed Tests}
\label{sect:detailedtests}

\subsection{Test Cycle LVV-C260 }

Open test cycle {\it \href{https://jira.lsstcorp.org/secure/Tests.jspa#/testrun/LVV-C260}{Data Management Acceptance Test Campaign, Fall 2023}} in Jira.

Test Cycle name: Data Management Acceptance Test Campaign, Fall 2023\\
Status: Done

This test cycle verifies a subset of
\href{https://lse-61.lsst.io/}{DMSR} (\citeds{LSE-61}) requirements in order to
verify their completion and readiness for LSST Operations (i.e., that
the requirements laid out in \citeds{LSE-61} have been met by the DM Systems).
Testing will use data products and artifacts from Data Preview 0.2
reprocessing of DESC DC2 data, Auxtel data, and other data products
housed at the U.S. Data Facility (USDF).

\subsubsection{Software Version/Baseline}
Primarily using Science Pipelines version 26 at the USDF.~

\subsubsection{Configuration}
Not provided.

\subsubsection{Test Cases in LVV-C260 Test Cycle}

\paragraph{ LVV-T146 - Verify implementation of DMS Initialization Component }\mbox{}\\

Version \textbf{1}.
Open  \href{https://jira.lsstcorp.org/secure/Tests.jspa#/testCase/LVV-T146}{\textit{ LVV-T146 } }
test case in Jira.

Demonstrate that all components of the DM system have a defined
deployment configuration within the DM deployment strategy

\textbf{ Preconditions}:\\


Final comment:\\


Detailed steps :

\begin{tabular}{p{2cm}}
\toprule
Step 1  \\ \hline
\end{tabular}
 Description \\
{\footnotesize
Inspect each service component to check if it has a deployment
configuration defined

}
\hdashrule[0.5ex]{\textwidth}{1pt}{3mm}
  Expected Result \\
{\footnotesize
All systems have a defined deployment configuration as part of the DM
deployment strategy\\[2\baselineskip]

}

\paragraph{ LVV-T1240 - Verify implementation of minimum astrometric standards per CCD }\mbox{}\\

Version \textbf{1}.
Open  \href{https://jira.lsstcorp.org/secure/Tests.jspa#/testCase/LVV-T1240}{\textit{ LVV-T1240 } }
test case in Jira.

Verify that each CCD in a processed dataset had its astrometric solution
determined by at least~\textbf{astrometricMinStandards = 5~}astrometric
standards.

\textbf{ Preconditions}:\\


Final comment:\\Test executed with science pipelines version w\_2023\_37 in the RSP
Notebook aspect at the USDF.\\[2\baselineskip]The executed notebook was
saved in the repository associated with this campaign's test report as
``notebooks/test\_LVV-T40\_T1240.ipynb''.


Detailed steps :

\begin{tabular}{p{2cm}}
\toprule
Step 1  \\ \hline
\end{tabular}
 Description \\
{\footnotesize
Identify an appropriate processed dataset for this test.

}
\hdashrule[0.5ex]{\textwidth}{1pt}{3mm}
  Expected Result \\
{\footnotesize
A dataset with Processed Visit Images.

}

\begin{tabular}{p{2cm}}
\toprule
Step 2  \\ \hline
\end{tabular}
 Description \\
{\footnotesize
Identify the path to the data repository, which we will refer to as
`DATA/path', then execute the following:

}
\hdashrule[0.5ex]{\textwidth}{1pt}{3mm}
  Example Code \\
{\footnotesize
\begin{verbatim}
from lsst.daf.butler import Butler
repo = 'Data/path'
collection = 'collection'
butler = Butler(repo, collections=collection)
\end{verbatim}

}
\hdashrule[0.5ex]{\textwidth}{1pt}{3mm}
  Expected Result \\
{\footnotesize
Butler repo available for reading.

}

\begin{tabular}{p{2cm}}
\toprule
Step 3  \\ \hline
\end{tabular}
 Description \\
{\footnotesize
Select a single visit from the dataset, and extract its calibration
data. For a subset of CCDs, check how many astrometric standards
contributed to the solution. Confirm that this number is at
least~\textbf{astrometricMinStandards = 5.}

}
\hdashrule[0.5ex]{\textwidth}{1pt}{3mm}
  Expected Result \\
{\footnotesize
At least \textbf{astrometricMinStandards} from each CCD\textbf{~}were
used in determining the WCS solution.

}

\paragraph{ LVV-T132 - Verify implementation of Pre-cursor and Real Data }\mbox{}\\

Version \textbf{1}.
Open  \href{https://jira.lsstcorp.org/secure/Tests.jspa#/testCase/LVV-T132}{\textit{ LVV-T132 } }
test case in Jira.

Demonstrate that pixel-oriented data from astronomical imaging cameras
(precursor or otherwise) can be processed using LSST Science Algorithms
and organized for access through the Data Butler Access Client. ~

\textbf{ Preconditions}:\\


Final comment:\\


Detailed steps :

\begin{tabular}{p{2cm}}
\toprule
Step 1  \\ \hline
\end{tabular}
 Description \\
{\footnotesize
Confirm that the CI jobs used to test DRP processing successfully run.
These jobs use precursor datasets from cameras other than LSST.

}
\hdashrule[0.5ex]{\textwidth}{1pt}{3mm}
  Expected Result \\
{\footnotesize

}

\begin{tabular}{p{2cm}}
\toprule
Step 2  \\ \hline
\end{tabular}
 Description \\
{\footnotesize
For the precursor dataset, instantiate the Butler, load the data
products, and confirm that they exist as expected.

}
\hdashrule[0.5ex]{\textwidth}{1pt}{3mm}
  Expected Result \\
{\footnotesize
Processed images, catalogs, calibration information, and other related
data products are present and accessible via the Butler.

}

\paragraph{ LVV-T62 - Verify implementation of Provide PSF for Coadded Images }\mbox{}\\

Version \textbf{2}.
Open  \href{https://jira.lsstcorp.org/secure/Tests.jspa#/testCase/LVV-T62}{\textit{ LVV-T62 } }
test case in Jira.

Verify that all coadd images produced by the DRP pipelines include a
model from which an image of the PSF at any point on the coadd can be
obtained.

\textbf{ Preconditions}:\\
Fully covered by preconditions for
\href{https://jira.lsstcorp.org/secure/Tests.jspa\#/testCase/LVV-T16}{LVV-T16}.

Final comment:\\Test executed with science pipelines version w\_2023\_34 in the RSP
Notebook aspect at the USDF.\\[2\baselineskip]The executed notebook was
saved in the repository associated with this campaign's test report as
``notebooks/test\_LVV-T62.ipynb''.


Detailed steps :

\begin{tabular}{p{2cm}}
\toprule
Step 1  \\ \hline
\end{tabular}
 Description \\
{\footnotesize
Identify a repo containing RC2 data with coadded images in multiple
filters.

}
\hdashrule[0.5ex]{\textwidth}{1pt}{3mm}
  Expected Result \\
{\footnotesize
Multi-band data that has been processed through the coaddition stage.

}

\begin{tabular}{p{2cm}}
\toprule
Step 2  \\ \hline
\end{tabular}
 Description \\
{\footnotesize
Identify the path to the data repository, which we will refer to as
`DATA/path', then execute the following:

}
\hdashrule[0.5ex]{\textwidth}{1pt}{3mm}
  Example Code \\
{\footnotesize
\begin{verbatim}
from lsst.daf.butler import Butler
repo = 'Data/path'
collection = 'collection'
butler = Butler(repo, collections=collection)
\end{verbatim}

}
\hdashrule[0.5ex]{\textwidth}{1pt}{3mm}
  Expected Result \\
{\footnotesize
Butler repo available for reading.

}

\begin{tabular}{p{2cm}}
\toprule
Step 3  \\ \hline
\end{tabular}
 Description \\
{\footnotesize
Load the exposures, then select Objects classified as point sources on
at least 10 different coadd images (including all bands). Evaluate the
PSF model at the positions of these Objects, and verify that subtracting
a scaled version of the PSF model from the processed visit image yields
residuals consistent with pure noise.

}
\hdashrule[0.5ex]{\textwidth}{1pt}{3mm}
  Expected Result \\
{\footnotesize
Images with the PSF model subtracted, leaving only residuals that are
consistent with being noise.

}

\paragraph{ LVV-T41 - Verify implementation of Generate PSF for Visit Images }\mbox{}\\

Version \textbf{1}.
Open  \href{https://jira.lsstcorp.org/secure/Tests.jspa#/testCase/LVV-T41}{\textit{ LVV-T41 } }
test case in Jira.

Verify that Processed Visit Images produced by the DRP and AP pipelines
are associated with a model from which one can obtain an image of the
PSF given a point on the image.

\textbf{ Preconditions}:\\


Final comment:\\Test executed with science pipelines version w\_2023\_37 in the RSP
Notebook aspect at the USDF.\\[2\baselineskip]The executed notebook was
saved in the repository associated with this campaign's test report as
``notebooks/test\_LVV-T41.ipynb''.


Detailed steps :

\begin{tabular}{p{2cm}}
\toprule
Step 1  \\ \hline
\end{tabular}
 Description \\
{\footnotesize
Identify a repo containing data with processed visit images in multiple
filters.

}
\hdashrule[0.5ex]{\textwidth}{1pt}{3mm}
  Expected Result \\
{\footnotesize

}

\begin{tabular}{p{2cm}}
\toprule
Step 2  \\ \hline
\end{tabular}
 Description \\
{\footnotesize
Identify the path to the data repository, which we will refer to as
`DATA/path', then execute the following:

}
\hdashrule[0.5ex]{\textwidth}{1pt}{3mm}
  Example Code \\
{\footnotesize
\begin{verbatim}
from lsst.daf.butler import Butler
repo = 'Data/path'
collection = 'collection'
butler = Butler(repo, collections=collection)
\end{verbatim}

}
\hdashrule[0.5ex]{\textwidth}{1pt}{3mm}
  Expected Result \\
{\footnotesize
Butler repo available for reading.

}

\begin{tabular}{p{2cm}}
\toprule
Step 3  \\ \hline
\end{tabular}
 Description \\
{\footnotesize
Select Objects classified as point sources on at least 10 different
processed visit images (including all bands). ~Evaluate the PSF model at
the positions of these Objects, and verify that subtracting a scaled
version of the PSF model from the processed visit image yields residuals
consistent with pure noise.

}
\hdashrule[0.5ex]{\textwidth}{1pt}{3mm}
  Expected Result \\
{\footnotesize
Images with the PSF model subtracted, leaving only residuals that are
consistent with being noise.

}

\paragraph{ LVV-T97 - Verify implementation of Uniqueness of IDs Across Data Releases }\mbox{}\\

Version \textbf{1}.
Open  \href{https://jira.lsstcorp.org/secure/Tests.jspa#/testCase/LVV-T97}{\textit{ LVV-T97 } }
test case in Jira.

Verify that the IDs of Objects, Sources, DIAObjects, and DIASources from
different Data Releases are unique.

\textbf{ Preconditions}:\\


Final comment:\\Executed at the USDF using the w\_2023\_43 version of the Science
Pipelines.\\[2\baselineskip]In addition to demonstrating that changing
the ``RELEASE\_ID'' results in unique IDs, we further examine the code
itself to demonstrate that the uniqueness of IDs is ensured by the way
the code is implemented.\\[2\baselineskip]The implementation of how the
64 bits of the Source/Object IDs are apportioned is in the
\texttt{\_IdGeneratorBits}\href{https://github.com/lsst/meas_base/blob/1bfaca56951770c88f4308da41de16d72ce40db9/python/lsst/meas/base/_id_generator.py\#L507-L537}{~class},
and especially its \texttt{\_\_post\_init\_\_}:\\

\begin{itemize}
\tightlist
\item
  it gets the maximum number of bits needed to pack the data ID (based
  on \texttt{\{visit,\ detector\}} for Source, and
  \texttt{\{tract,\ patch\}} for Object) and turns that into the number
  of distinct data IDs it could hold \texttt{(n\_data\_ids)};
\item
  it multiplies that with the \texttt{n\_releases} option to identify
  how many ``upper'' bits need to be reserved for
  \texttt{n\_data\_ids*n\_releases} ``catalog\_ids'', and sets
  \texttt{n\_counters} to be the number of values remaining in the
  Source or Object 64-bit ID to count sources or objects within one
  image.
\end{itemize}

Because some interfaces historically wanted to count bits rather than
just multiply integers, there is a mix of direct multiplication and log2
addition. But you can see the result most clearly in
\texttt{FullIdGenerator.arange} and
\texttt{FullIdGenerator.catalog\_id}; a full Source or Object ID is:\\

\begin{verbatim}
id = counter + n_counters * catalog_id
\end{verbatim}

or\\

\begin{verbatim}
id = counter + n_counters * (packed_data_id + n_data_ids * release_id)
\end{verbatim}

and hence the releases will get distinct IDs as long as counter values
are less than \texttt{n\_counters} and the packed data IDs are less than
\texttt{n\_data\_ids}. (And there are several checks throughout the file
for those criteria).


Detailed steps :

\begin{tabular}{p{2cm}}
\toprule
Step 1  \\ \hline
\end{tabular}
 Description \\
{\footnotesize
Identify an appropriate precursor dataset to be processed through Data
Release Production.

}
\hdashrule[0.5ex]{\textwidth}{1pt}{3mm}
  Expected Result \\
{\footnotesize

}

\begin{tabular}{p{2cm}}
\toprule
Step 2  \\ \hline
\end{tabular}
 Description \\
{\footnotesize
Process data with the Data Release Production payload, starting from raw
science images and generating science data products, placing them in the
Data Backbone.

}
\hdashrule[0.5ex]{\textwidth}{1pt}{3mm}
  Expected Result \\
{\footnotesize

}

\begin{tabular}{p{2cm}}
\toprule
Step 3  \\ \hline
\end{tabular}
 Description \\
{\footnotesize
Identify the path to the data repository, which we will refer to as
`DATA/path', then execute the following:

}
\hdashrule[0.5ex]{\textwidth}{1pt}{3mm}
  Example Code \\
{\footnotesize
\begin{verbatim}
from lsst.daf.butler import Butler
repo = 'Data/path'
collection = 'collection'
butler = Butler(repo, collections=collection)
\end{verbatim}

}
\hdashrule[0.5ex]{\textwidth}{1pt}{3mm}
  Expected Result \\
{\footnotesize
Butler repo available for reading.

}

\begin{tabular}{p{2cm}}
\toprule
Step 4  \\ \hline
\end{tabular}
 Description \\
{\footnotesize
After running the DRP payload multiple times, load the resulting data
products (both data release and prompt products) using the Butler.

}
\hdashrule[0.5ex]{\textwidth}{1pt}{3mm}
  Expected Result \\
{\footnotesize
Multiple datasets resulting from processing of the same input data.

}

\begin{tabular}{p{2cm}}
\toprule
Step 5  \\ \hline
\end{tabular}
 Description \\
{\footnotesize
Inspect the IDs in the multiple data products and confirm that all IDs
are unique.

}
\hdashrule[0.5ex]{\textwidth}{1pt}{3mm}
  Expected Result \\
{\footnotesize
No IDs are repeated between multiple processings of the identical input
dataset.

}

\paragraph{ LVV-T1946 - Verify implementation of measurements in catalogs from coadds }\mbox{}\\

Version \textbf{1}.
Open  \href{https://jira.lsstcorp.org/secure/Tests.jspa#/testCase/LVV-T1946}{\textit{ LVV-T1946 } }
test case in Jira.

Verify that source measurements in catalogs containing measurements from
coadd images are in flux units.

\textbf{ Preconditions}:\\


Final comment:\\This test case can be executed by running the script test\_LVV-T1946.py,
which is available in the test report github repository's ``scripts/''
directory.\\[2\baselineskip]The tests that confirm fluxes have
``reasonable'' values are checking that the fluxes, if converted to
magnitudes, would result in a magnitude fainter than -5.


Detailed steps :

\begin{tabular}{p{2cm}}
\toprule
Step 1  \\ \hline
\end{tabular}
 Description \\
{\footnotesize
Identify the path to the data repository, which we will refer to as
`DATA/path', then execute the following:

}
\hdashrule[0.5ex]{\textwidth}{1pt}{3mm}
  Example Code \\
{\footnotesize
\begin{verbatim}
from lsst.daf.butler import Butler
repo = 'Data/path'
collection = 'collection'
butler = Butler(repo, collections=collection)
\end{verbatim}

}
\hdashrule[0.5ex]{\textwidth}{1pt}{3mm}
  Expected Result \\
{\footnotesize
Butler repo available for reading.

}

\begin{tabular}{p{2cm}}
\toprule
Step 2  \\ \hline
\end{tabular}
 Description \\
{\footnotesize
Identify and read an appropriate processed precursor dataset containing
coadds with the Butler.

}
\hdashrule[0.5ex]{\textwidth}{1pt}{3mm}
  Expected Result \\
{\footnotesize

}

\begin{tabular}{p{2cm}}
\toprule
Step 3  \\ \hline
\end{tabular}
 Description \\
{\footnotesize
Verify that the coadd catalog provides measurements in flux units.

}
\hdashrule[0.5ex]{\textwidth}{1pt}{3mm}
  Expected Result \\
{\footnotesize
Confirmation of measurements in catalogs encoded in flux units.

}

\paragraph{ LVV-T1947 - Verify implementation of measurements in catalogs from difference images }\mbox{}\\

Version \textbf{1}.
Open  \href{https://jira.lsstcorp.org/secure/Tests.jspa#/testCase/LVV-T1947}{\textit{ LVV-T1947 } }
test case in Jira.

Verify that source measurements in catalogs containing measurements from
difference images are in flux units.

\textbf{ Preconditions}:\\


Final comment:\\This test case can be executed by running the scripts
test\_LVV-T1947\_DiaSource.py,
~test\_LVV-T1947\_forcedSourceOnDiaObject.py, and
test\_LVV-T1947\_DiaObject.py, which are available in the test report
github repository's ``scripts/'' directory.\\[2\baselineskip]The tests
that confirm fluxes have ``reasonable'' values are checking that the
fluxes, if converted to magnitudes, would result in a magnitude fainter
than -5.


Detailed steps :

\begin{tabular}{p{2cm}}
\toprule
Step 1  \\ \hline
\end{tabular}
 Description \\
{\footnotesize
Identify the path to the data repository, which we will refer to as
`DATA/path', then execute the following:

}
\hdashrule[0.5ex]{\textwidth}{1pt}{3mm}
  Example Code \\
{\footnotesize
\begin{verbatim}
from lsst.daf.butler import Butler
repo = 'Data/path'
collection = 'collection'
butler = Butler(repo, collections=collection)
\end{verbatim}

}
\hdashrule[0.5ex]{\textwidth}{1pt}{3mm}
  Expected Result \\
{\footnotesize
Butler repo available for reading.

}

\begin{tabular}{p{2cm}}
\toprule
Step 2  \\ \hline
\end{tabular}
 Description \\
{\footnotesize
Identify and read an appropriate processed precursor dataset containing
difference images with the Butler.

}
\hdashrule[0.5ex]{\textwidth}{1pt}{3mm}
  Expected Result \\
{\footnotesize

}

\begin{tabular}{p{2cm}}
\toprule
Step 3  \\ \hline
\end{tabular}
 Description \\
{\footnotesize
Verify that the difference image source catalog provides measurements in
flux units.

}
\hdashrule[0.5ex]{\textwidth}{1pt}{3mm}
  Expected Result \\
{\footnotesize
Confirmation of measurements in catalogs encoded in flux units.

}

\paragraph{ LVV-T28 - Verify implementation of measurements in catalogs from PVIs }\mbox{}\\

Version \textbf{1}.
Open  \href{https://jira.lsstcorp.org/secure/Tests.jspa#/testCase/LVV-T28}{\textit{ LVV-T28 } }
test case in Jira.

Verify that source measurements in catalogs containing measurements from
processed visit images are in flux units.

\textbf{ Preconditions}:\\


Final comment:\\This test case can be executed by running the scripts test\_LVV-T28.py,
test\_LVV-T28\_forcedSource.py, and test\_LVV-T28\_DC2.py, which are
available in the test report github repository's ``scripts/''
directory.\\[2\baselineskip]The tests that confirm fluxes have
``reasonable'' values are checking that the fluxes, if converted to
magnitudes, would result in a magnitude fainter than -5.


Detailed steps :

\begin{tabular}{p{2cm}}
\toprule
Step 1  \\ \hline
\end{tabular}
 Description \\
{\footnotesize
Identify the path to the data repository, which we will refer to as
`DATA/path', then execute the following:

}
\hdashrule[0.5ex]{\textwidth}{1pt}{3mm}
  Example Code \\
{\footnotesize
\begin{verbatim}
from lsst.daf.butler import Butler
repo = 'Data/path'
collection = 'collection'
butler = Butler(repo, collections=collection)
\end{verbatim}

}
\hdashrule[0.5ex]{\textwidth}{1pt}{3mm}
  Expected Result \\
{\footnotesize
Butler repo available for reading.

}

\begin{tabular}{p{2cm}}
\toprule
Step 2  \\ \hline
\end{tabular}
 Description \\
{\footnotesize
Identify and read an appropriate repo with processed {RC2}⁠ ~precursor
data containing coadds with the Butler.

}
\hdashrule[0.5ex]{\textwidth}{1pt}{3mm}
  Expected Result \\
{\footnotesize

}

\begin{tabular}{p{2cm}}
\toprule
Step 3  \\ \hline
\end{tabular}
 Description \\
{\footnotesize
Verify that the single-visit catalog provides measurements in flux
units.

}
\hdashrule[0.5ex]{\textwidth}{1pt}{3mm}
  Expected Result \\
{\footnotesize
Confirmation of measurements in catalogs encoded in flux units.

}

\begin{tabular}{p{2cm}}
\toprule
Step 4  \\ \hline
\end{tabular}
 Description \\
{\footnotesize
Identify the path to the data repository, which we will refer to as
`DATA/path', then execute the following:

}
\hdashrule[0.5ex]{\textwidth}{1pt}{3mm}
  Example Code \\
{\footnotesize
\begin{verbatim}
from lsst.daf.butler import Butler
repo = 'Data/path'
collection = 'collection'
butler = Butler(repo, collections=collection)
\end{verbatim}

}
\hdashrule[0.5ex]{\textwidth}{1pt}{3mm}
  Expected Result \\
{\footnotesize
Butler repo available for reading.

}

\begin{tabular}{p{2cm}}
\toprule
Step 5  \\ \hline
\end{tabular}
 Description \\
{\footnotesize
Identify and read an appropriate repo with processed {DC2}⁠ ~precursor
data containing coadds with the Butler.

}
\hdashrule[0.5ex]{\textwidth}{1pt}{3mm}
  Expected Result \\
{\footnotesize

}

\begin{tabular}{p{2cm}}
\toprule
Step 6  \\ \hline
\end{tabular}
 Description \\
{\footnotesize
Verify that the single-visit catalog provides measurements in flux
units.

}
\hdashrule[0.5ex]{\textwidth}{1pt}{3mm}
  Expected Result \\
{\footnotesize
Confirmation of measurements in catalogs encoded in flux units.

}

\paragraph{ LVV-T142 - Verify implementation of Production Fault Tolerance }\mbox{}\\

Version \textbf{1}.
Open  \href{https://jira.lsstcorp.org/secure/Tests.jspa#/testCase/LVV-T142}{\textit{ LVV-T142 } }
test case in Jira.

Demonstrate production systems report faults in pipeline executions and
that system is able to recover. ~Where recovery can mean the ability to
provide production artifacts for examination, return production elements
ready for subsequent use, and/or reset and repeat production attempts.

\textbf{ Preconditions}:\\


Final comment:\\Executed at the USDF using the w\_2023\_43 version of the Science
Pipelines.


Detailed steps :

\begin{tabular}{p{2cm}}
\toprule
Step 1  \\ \hline
\end{tabular}
 Description \\
{\footnotesize
Create a HTCondor Pool at USDF

}
\hdashrule[0.5ex]{\textwidth}{1pt}{3mm}
  Example Code \\
{\footnotesize
allocateNodes.py -c 8 -m 4-00:00:00 -q roma,milano -g 900 s3df

}
\hdashrule[0.5ex]{\textwidth}{1pt}{3mm}
  Expected Result \\
{\footnotesize

}

\begin{tabular}{p{2cm}}
\toprule
Step 2  \\ \hline
\end{tabular}
 Description \\
{\footnotesize
Submit a RC2-subset run using bps

}
\hdashrule[0.5ex]{\textwidth}{1pt}{3mm}
  Example Code \\
{\footnotesize
bps submit --output u/username/collectionname
\$\{DRP\_PIPE\_DIR\}/drp\_pipe/pipelines/HSC/DRP-RC2\_subset.yaml

}
\hdashrule[0.5ex]{\textwidth}{1pt}{3mm}
  Expected Result \\
{\footnotesize

}

\begin{tabular}{p{2cm}}
\toprule
Step 3  \\ \hline
\end{tabular}
 Description \\
{\footnotesize
Cancel the job while it is in progress

}
\hdashrule[0.5ex]{\textwidth}{1pt}{3mm}
  Example Code \\
{\footnotesize
bps cancel {[}job id{]}

}
\hdashrule[0.5ex]{\textwidth}{1pt}{3mm}
  Expected Result \\
{\footnotesize

}

\begin{tabular}{p{2cm}}
\toprule
Step 4  \\ \hline
\end{tabular}
 Description \\
{\footnotesize
Check that some output products are missing

}
\hdashrule[0.5ex]{\textwidth}{1pt}{3mm}
  Example Code \\
{\footnotesize
butler query-datasets /repo/main --collections u/username/collectionname

}
\hdashrule[0.5ex]{\textwidth}{1pt}{3mm}
  Expected Result \\
{\footnotesize

}

\begin{tabular}{p{2cm}}
\toprule
Step 5  \\ \hline
\end{tabular}
 Description \\
{\footnotesize
Submit a new job to finish the tasks that were not completed

}
\hdashrule[0.5ex]{\textwidth}{1pt}{3mm}
  Example Code \\
{\footnotesize
bps submit --output u/username/collectionname \$\{DRP\_PIPE\_DIR\}
drp\_pipe/pipelines/HSC/DRP-RC2\_subset.yaml

}
\hdashrule[0.5ex]{\textwidth}{1pt}{3mm}
  Expected Result \\
{\footnotesize

}

\begin{tabular}{p{2cm}}
\toprule
Step 6  \\ \hline
\end{tabular}
 Description \\
{\footnotesize
Confirm that all expected files are now present

}
\hdashrule[0.5ex]{\textwidth}{1pt}{3mm}
  Example Code \\
{\footnotesize
butler query-datasets /repo/main --collections u/username/collectionname

}
\hdashrule[0.5ex]{\textwidth}{1pt}{3mm}
  Expected Result \\
{\footnotesize

}

\paragraph{ LVV-T1748 - Verify calculation of median error in absolute position for RA, Dec axes }\mbox{}\\

Version \textbf{1}.
Open  \href{https://jira.lsstcorp.org/secure/Tests.jspa#/testCase/LVV-T1748}{\textit{ LVV-T1748 } }
test case in Jira.

Verify that the DM system has provided the code to calculate the median
error in absolute position for each axis, RA and DEC, and assess whether
it meets the requirement that it shall be less than \textbf{AA1 = 50
milliarcseconds}.

\textbf{ Preconditions}:\\


Final comment:\\


Detailed steps :

\begin{tabular}{p{2cm}}
\toprule
Step 1  \\ \hline
\end{tabular}
 Description \\
{\footnotesize
Identify a dataset containing processed data.

}
\hdashrule[0.5ex]{\textwidth}{1pt}{3mm}
  Expected Result \\
{\footnotesize
A dataset that has been ingested into a Butler repository.

}

\begin{tabular}{p{2cm}}
\toprule
Step 2  \\ \hline
\end{tabular}
 Description \\
{\footnotesize
The `path` that you will use depends on where you are running the
science pipelines. Options:\\[2\baselineskip]

\begin{itemize}
\tightlist
\item
  local (newinstall.sh - based
  install):{[}path\_to\_installation{]}/loadLSST.bash
\item
  development cluster (``lsst-dev''):
  /software/lsstsw/stack/loadLSST.bash
\item
  LSP Notebook aspect (from a terminal):
  /opt/lsst/software/stack/loadLSST.bash
\end{itemize}

From the command line, execute the commands below in the example
code:\\[2\baselineskip]

}
\hdashrule[0.5ex]{\textwidth}{1pt}{3mm}
  Example Code \\
{\footnotesize
source `path`\\
setup lsst\_distrib

}
\hdashrule[0.5ex]{\textwidth}{1pt}{3mm}
  Expected Result \\
{\footnotesize
Science pipeline software is available for use. If additional packages
are needed (for example, `obs' packages such as `obs\_subaru`), then
additional `setup` commands will be necessary.\\[2\baselineskip]To check
versions in use, type:\\
eups list -s

}

\begin{tabular}{p{2cm}}
\toprule
Step 3  \\ \hline
\end{tabular}
 Description \\
{\footnotesize
Execute `analysis\_tools` on a repository containing processed data.
Identify the path to the data, which we will call `DATA/path', then
execute something similar to the following (with paths, datasets, and
flags replaced or additionally specified as needed):

}
\hdashrule[0.5ex]{\textwidth}{1pt}{3mm}
  Example Code \\
{\footnotesize
pipetask --long-log run -j 2 -b DATA/path/butler.yaml
--register-dataset-types -p
\$ANALYSIS\_TOOLS\_DIR/pipelines/matchedVisitQualityCore.yaml -d ``band
in ('g', `r', `i') AND tract=9813 AND skymap='hsc\_rings\_v1' AND
instrument='HSC''' --output u/username/atools\_metrics -i
HSC/runs/RC2/w\_2023\_36 --instrument lsst.obs.subaru.HyperSuprimeCam
2\textgreater{}\&1 \textbar{} tee w36\_2023\_tract9813\_atools.txt

}
\hdashrule[0.5ex]{\textwidth}{1pt}{3mm}
  Expected Result \\
{\footnotesize
The output collection (in this case, ``u/username/atools\_metrics'')
containing metric measurements and any associated extras and metadata is
available via the butler.

}

\begin{tabular}{p{2cm}}
\toprule
Step 4  \\ \hline
\end{tabular}
 Description \\
{\footnotesize
Confirm that the metric AA1 has been calculated, and that its values are
reasonable.

}
\hdashrule[0.5ex]{\textwidth}{1pt}{3mm}
  Expected Result \\
{\footnotesize
A JSON file (and/or a report generated from that JSON file)
demonstrating that AA1 has been calculated.

}

\paragraph{ LVV-T1759 - Verify that the repeatability outlier limit for isolated bright
non-saturated point sources in the g, r, and i filters (PA2gri) can be
applied. }\mbox{}\\

Version \textbf{1}.
Open  \href{https://jira.lsstcorp.org/secure/Tests.jspa#/testCase/LVV-T1759}{\textit{ LVV-T1759 } }
test case in Jira.

Verify that the DM system has provided the code to apply the
repeatability outlier limit for isolated bright non-saturated point
sources in the g, r, and i filters(PA2gri) to to computed values of the
PF1 metric.

\textbf{ Preconditions}:\\


Final comment:\\This test used a modified version of the analysis\_tools pipeline
``matchedVisitQualityCore.yaml.''


Detailed steps :

\begin{tabular}{p{2cm}}
\toprule
Step 1  \\ \hline
\end{tabular}
 Description \\
{\footnotesize
Identify a dataset containing at least one field in each of the g, r,
and i filters with multiple overlapping visits.

}
\hdashrule[0.5ex]{\textwidth}{1pt}{3mm}
  Expected Result \\
{\footnotesize
A dataset that has been ingested into a Butler repository.

}

\begin{tabular}{p{2cm}}
\toprule
Step 2  \\ \hline
\end{tabular}
 Description \\
{\footnotesize
The `path` that you will use depends on where you are running the
science pipelines. Options:\\[2\baselineskip]

\begin{itemize}
\tightlist
\item
  local (newinstall.sh - based
  install):{[}path\_to\_installation{]}/loadLSST.bash
\item
  development cluster (``lsst-dev''):
  /software/lsstsw/stack/loadLSST.bash
\item
  LSP Notebook aspect (from a terminal):
  /opt/lsst/software/stack/loadLSST.bash
\end{itemize}

From the command line, execute the commands below in the example
code:\\[2\baselineskip]

}
\hdashrule[0.5ex]{\textwidth}{1pt}{3mm}
  Example Code \\
{\footnotesize
source `path`\\
setup lsst\_distrib

}
\hdashrule[0.5ex]{\textwidth}{1pt}{3mm}
  Expected Result \\
{\footnotesize
Science pipeline software is available for use. If additional packages
are needed (for example, `obs' packages such as `obs\_subaru`), then
additional `setup` commands will be necessary.\\[2\baselineskip]To check
versions in use, type:\\
eups list -s

}

\begin{tabular}{p{2cm}}
\toprule
Step 3  \\ \hline
\end{tabular}
 Description \\
{\footnotesize
Execute `analysis\_tools` on a repository containing processed data.
Identify the path to the data, which we will call `DATA/path', then
execute something similar to the following (with paths, datasets, and
flags replaced or additionally specified as needed):

}
\hdashrule[0.5ex]{\textwidth}{1pt}{3mm}
  Example Code \\
{\footnotesize
pipetask --long-log run -j 2 -b DATA/path/butler.yaml
--register-dataset-types -p
\$ANALYSIS\_TOOLS\_DIR/pipelines/matchedVisitQualityCore.yaml -d ``band
in ('g', `r', `i') AND tract=9813 AND skymap='hsc\_rings\_v1' AND
instrument='HSC''' --output u/username/atools\_metrics -i
HSC/runs/RC2/w\_2023\_36 --instrument lsst.obs.subaru.HyperSuprimeCam
2\textgreater{}\&1 \textbar{} tee w36\_2023\_tract9813\_atools.txt

}
\hdashrule[0.5ex]{\textwidth}{1pt}{3mm}
  Expected Result \\
{\footnotesize
The output collection (in this case, ``u/username/atools\_metrics'')
containing metric measurements and any associated extras and metadata is
available via the butler.

}

\begin{tabular}{p{2cm}}
\toprule
Step 4  \\ \hline
\end{tabular}
 Description \\
{\footnotesize
Confirm that the PA2gri threshold has been applied to the assessment of
the computed values of PF1 for filters g,r,i.

}
\hdashrule[0.5ex]{\textwidth}{1pt}{3mm}
  Expected Result \\
{\footnotesize
A JSON file (and/or a report generated from that JSON file)
demonstrating that PF1 has been calculated (and that it used the
requested threshold value of PA2gri).

}

\begin{tabular}{p{2cm}}
\toprule
Step 5  \\ \hline
\end{tabular}
 Description \\
{\footnotesize
Change the value of the PA2 threshold in the pipeline yaml for
analysis\_tools, then rerun analysis\_tools

}
\hdashrule[0.5ex]{\textwidth}{1pt}{3mm}
  Expected Result \\
{\footnotesize

}

\begin{tabular}{p{2cm}}
\toprule
Step 6  \\ \hline
\end{tabular}
 Description \\
{\footnotesize
Confirm that the new PA2 threshold has been applied when computing PF1.

}
\hdashrule[0.5ex]{\textwidth}{1pt}{3mm}
  Expected Result \\
{\footnotesize
A JSON file (and/or a report generated from that JSON file)
demonstrating that PF1 has been calculated (and that it used the
requested threshold value of PA2gri).

}

\paragraph{ LVV-T1758 - Verify that the repeatability outlier limit for isolated bright
non-saturated point sources in the u, z, and y filters (PA2uzy) can be
applied. }\mbox{}\\

Version \textbf{1}.
Open  \href{https://jira.lsstcorp.org/secure/Tests.jspa#/testCase/LVV-T1758}{\textit{ LVV-T1758 } }
test case in Jira.

Verify that the DM system has provided the code to apply the
repeatability outlier limit for isolated bright non-saturated point
sources in the u, z, and y filters(PA2uzy) to computed values of the PF1
metric.

\textbf{ Preconditions}:\\


Final comment:\\Note that because we do not have access to u-band data, this test was
performed for only y- and z-band. The steps would be unchanged for
u-band data.


Detailed steps :

\begin{tabular}{p{2cm}}
\toprule
Step 1  \\ \hline
\end{tabular}
 Description \\
{\footnotesize
Identify a dataset containing at least one field in each of the u, z,
and y filters with multiple overlapping visits.

}
\hdashrule[0.5ex]{\textwidth}{1pt}{3mm}
  Expected Result \\
{\footnotesize
A dataset that has been ingested into a Butler repository.

}

\begin{tabular}{p{2cm}}
\toprule
Step 2  \\ \hline
\end{tabular}
 Description \\
{\footnotesize
The `path` that you will use depends on where you are running the
science pipelines. Options:\\[2\baselineskip]

\begin{itemize}
\tightlist
\item
  local (newinstall.sh - based
  install):{[}path\_to\_installation{]}/loadLSST.bash
\item
  development cluster (``lsst-dev''):
  /software/lsstsw/stack/loadLSST.bash
\item
  LSP Notebook aspect (from a terminal):
  /opt/lsst/software/stack/loadLSST.bash
\end{itemize}

From the command line, execute the commands below in the example
code:\\[2\baselineskip]

}
\hdashrule[0.5ex]{\textwidth}{1pt}{3mm}
  Example Code \\
{\footnotesize
source `path`\\
setup lsst\_distrib

}
\hdashrule[0.5ex]{\textwidth}{1pt}{3mm}
  Expected Result \\
{\footnotesize
Science pipeline software is available for use. If additional packages
are needed (for example, `obs' packages such as `obs\_subaru`), then
additional `setup` commands will be necessary.\\[2\baselineskip]To check
versions in use, type:\\
eups list -s

}

\begin{tabular}{p{2cm}}
\toprule
Step 3  \\ \hline
\end{tabular}
 Description \\
{\footnotesize
Execute `analysis\_tools` on a repository containing processed data.
Identify the path to the data, which we will call `DATA/path', then
execute something similar to the following (with paths, datasets, and
flags replaced or additionally specified as needed):

}
\hdashrule[0.5ex]{\textwidth}{1pt}{3mm}
  Example Code \\
{\footnotesize
pipetask --long-log run -j 2 -b DATA/path/butler.yaml
--register-dataset-types -p
\$ANALYSIS\_TOOLS\_DIR/pipelines/matchedVisitQualityCore.yaml -d ``band
in ('g', `r', `i') AND tract=9813 AND skymap='hsc\_rings\_v1' AND
instrument='HSC''' --output u/username/atools\_metrics -i
HSC/runs/RC2/w\_2023\_36 --instrument lsst.obs.subaru.HyperSuprimeCam
2\textgreater{}\&1 \textbar{} tee w36\_2023\_tract9813\_atools.txt

}
\hdashrule[0.5ex]{\textwidth}{1pt}{3mm}
  Expected Result \\
{\footnotesize
The output collection (in this case, ``u/username/atools\_metrics'')
containing metric measurements and any associated extras and metadata is
available via the butler.

}

\begin{tabular}{p{2cm}}
\toprule
Step 4  \\ \hline
\end{tabular}
 Description \\
{\footnotesize
Confirm that the PA2uzy threshold has been applied to the assessment of
the computed values of PF1 for filters u,z,y.

}
\hdashrule[0.5ex]{\textwidth}{1pt}{3mm}
  Expected Result \\
{\footnotesize
A JSON file (and/or a report generated from that JSON file)
demonstrating that PF1 has been calculated (and that it used the
requested PA2uzy threshold).

}

\begin{tabular}{p{2cm}}
\toprule
Step 5  \\ \hline
\end{tabular}
 Description \\
{\footnotesize
Change the value of the PA2 threshold in the pipeline yaml for
analysis\_tools, then rerun analysis\_tools

}
\hdashrule[0.5ex]{\textwidth}{1pt}{3mm}
  Expected Result \\
{\footnotesize

}

\begin{tabular}{p{2cm}}
\toprule
Step 6  \\ \hline
\end{tabular}
 Description \\
{\footnotesize
Confirm that the new PA2 threshold has been applied when computing PF1.

}
\hdashrule[0.5ex]{\textwidth}{1pt}{3mm}
  Expected Result \\
{\footnotesize
A JSON file (and/or a report generated from that JSON file)
demonstrating that PF1 has been calculated (and that it used the
requested threshold value of PA2gri).

}

\paragraph{ LVV-T149 - Verify implementation of Catalog Queries }\mbox{}\\

Version \textbf{1}.
Open  \href{https://jira.lsstcorp.org/secure/Tests.jspa#/testCase/LVV-T149}{\textit{ LVV-T149 } }
test case in Jira.

Verify that SQL, or a similar structured language, can be used to query
catalogs.

\textbf{ Preconditions}:\\
An operational QSERV database that has been verified via
\href{https://jira.lsstcorp.org/secure/Tests.jspa\#/testCase/LVV-T1085}{LVV-T1085}
and
\href{https://jira.lsstcorp.org/secure/Tests.jspa\#/testCase/LVV-T1086}{LVV-T1086}
and
\href{https://jira.lsstcorp.org/secure/Tests.jspa\#/testCase/LVV-T1087}{LVV-T1087}.

Final comment:\\Executed using the IDF Notebook, Portal, and API aspects. For the
notebook execution, we used science pipelines version w\_2023\_34.


Detailed steps :

\begin{tabular}{p{2cm}}
\toprule
Step 1  \\ \hline
\end{tabular}
 Description \\
{\footnotesize
Execute a simple query (for example, the one below) and confirm that it
returns the expected result.

}
\hdashrule[0.5ex]{\textwidth}{1pt}{3mm}
  Example Code \\
{\footnotesize
SELECT * FROM dp02\_dc2\_catalogs.Object as obj WHERE
CONTAINS(POINT('ICRS', obj.coord\_ra, obj.coord\_dec), CIRCLE('ICRS',
62.0, -37.0, 0.10)) = 1

}
\hdashrule[0.5ex]{\textwidth}{1pt}{3mm}
  Expected Result \\
{\footnotesize
A catalog of objects satisfying the specified constraints. The catalog
should contain 26,115 results.

}

\begin{tabular}{p{2cm}}
\toprule
Step 2  \\ \hline
\end{tabular}
 Description \\
{\footnotesize
Repeat the query from all available access routes (e.g., an external VO
client, the Science Platform query tool, and from within the Notebook
Aspect), confirming in each case that the results are as expected.

}
\hdashrule[0.5ex]{\textwidth}{1pt}{3mm}
  Expected Result \\
{\footnotesize

}

\paragraph{ LVV-T40 - Verify implementation of Generate WCS for Visit Images }\mbox{}\\

Version \textbf{1}.
Open  \href{https://jira.lsstcorp.org/secure/Tests.jspa#/testCase/LVV-T40}{\textit{ LVV-T40 } }
test case in Jira.

Verify that Processed Visit Images produced by the AP and DRP pipelines
include FITS WCS accurate to specified \textbf{astrometricAccuracy} over
the bounds of the image.

\textbf{ Preconditions}:\\


Final comment:\\Test executed with science pipelines version w\_2023\_37 in the RSP
Notebook aspect at the USDF.\\[2\baselineskip]The executed notebook was
saved in the repository associated with this campaign's test report as
``notebooks/test\_LVV-T40\_T1240.ipynb''.


Detailed steps :

\begin{tabular}{p{2cm}}
\toprule
Step 1  \\ \hline
\end{tabular}
 Description \\
{\footnotesize
Identify an appropriate repo containing processed HSC data for this
test.

}
\hdashrule[0.5ex]{\textwidth}{1pt}{3mm}
  Expected Result \\
{\footnotesize
A dataset with Processed Visit Images available.

}

\begin{tabular}{p{2cm}}
\toprule
Step 2  \\ \hline
\end{tabular}
 Description \\
{\footnotesize
Identify the path to the data repository, which we will refer to as
`DATA/path', then execute the following:

}
\hdashrule[0.5ex]{\textwidth}{1pt}{3mm}
  Example Code \\
{\footnotesize
\begin{verbatim}
from lsst.daf.butler import Butler
repo = 'Data/path'
collection = 'collection'
butler = Butler(repo, collections=collection)
\end{verbatim}

}
\hdashrule[0.5ex]{\textwidth}{1pt}{3mm}
  Expected Result \\
{\footnotesize
Butler repo available for reading.

}

\begin{tabular}{p{2cm}}
\toprule
Step 3  \\ \hline
\end{tabular}
 Description \\
{\footnotesize
Select a single visit from the dataset, and extract its WCS object and
the source list.

}
\hdashrule[0.5ex]{\textwidth}{1pt}{3mm}
  Expected Result \\
{\footnotesize
A table containing detected sources, and a WCS object associated with
that catalog.

}

\begin{tabular}{p{2cm}}
\toprule
Step 4  \\ \hline
\end{tabular}
 Description \\
{\footnotesize
Confirm that each CCD within the visit image contains at
least~\textbf{astrometricMinStandards~}astrometric standards that were
used in deriving the astrometric solution.

}
\hdashrule[0.5ex]{\textwidth}{1pt}{3mm}
  Expected Result \\
{\footnotesize
At least \textbf{astrometricMinStandards} from each CCD\textbf{~}were
used in determining the WCS solution.

}

\begin{tabular}{p{2cm}}
\toprule
Step 5  \\ \hline
\end{tabular}
 Description \\
{\footnotesize
Starting from the XY pixel coordinates of the sources, apply the WCS to
obtain RA, Dec coordinates.\\[2\baselineskip]

}
\hdashrule[0.5ex]{\textwidth}{1pt}{3mm}
  Expected Result \\
{\footnotesize
A list of RA, Dec coordinates for all sources in the catalog.

}

\begin{tabular}{p{2cm}}
\toprule
Step 6  \\ \hline
\end{tabular}
 Description \\
{\footnotesize
We will assume that Gaia provides a source of ``truth.'' Match the
source list to Gaia DR3, and calculate the positional offset between the
test data and the Gaia catalog.

}
\hdashrule[0.5ex]{\textwidth}{1pt}{3mm}
  Expected Result \\
{\footnotesize
A matched catalog of sources in common between the test source list and
Gaia DR3.

}

\begin{tabular}{p{2cm}}
\toprule
Step 7  \\ \hline
\end{tabular}
 Description \\
{\footnotesize
Apply appropriate cuts to extract the optimal dataset for comparison,
then calculate statistics (median, 1-sigma range, etc.; also plot a
histogram) of the offsets in milliarcseconds. Confirm that the offset is
less than \textbf{astrometricAccuracy}.

}
\hdashrule[0.5ex]{\textwidth}{1pt}{3mm}
  Expected Result \\
{\footnotesize
Histogram and relevant statistics needed to confirm that the WCS
transformation is accurate.

}

\begin{tabular}{p{2cm}}
\toprule
Step 8  \\ \hline
\end{tabular}
 Description \\
{\footnotesize
Repeat Step 5, but for subregions of the image, to confirm that the
accuracy criterion is met at all positions.

}
\hdashrule[0.5ex]{\textwidth}{1pt}{3mm}
  Expected Result \\
{\footnotesize
\textbf{astrometricAccuracy~}requirement is met over the entire image.

}




% This appendix is put in as part of the template. You may edit and add to it.
% It is not overwritten by Docsteady.

\newpage
\appendix
\section{Documentation}
The verification process is defined in \citeds{LSE-160}.
The use of Docsteady to format Jira information in various test and planing documents is
described in \citeds{DMTN-140} and practical commands are given in \citeds{DMTN-178}.

\section{Acronyms used in this document}\label{sec:acronyms}
\input{acronyms.tex}

\newpage

% Uncomment this if Docsteady makes you additional appendix
%\input{DMTR-401.appendix.tex}

\end{document}
